\documentclass{article}

	\usepackage{hevea}

	\title{The SDF Parse Table Format}
	\author{Karl Trygve Kalleberg}
	
	\newcommand{\cc}[1]{\texttt{#1}}
	\newcommand{\tp}[1]{\texttt{#1}}
	\newcommand{\code}[1]{\texttt{#1}}

\begin{document}

\section{The SGLR Parse Table Format}

	The version 4 (now outdated) parse table format is explained here. The newer
	formats are very similar, but some details are different. For now, please
	refer to the source code for details.

	\code{parse-table(<term>,<term>,<term>,states([<list>]),priorities(<term>))}

	\begin{itemize}
      \item The first \code{<term>} is a version number, currently 4.
      \item The second \code{<term>} is the number of the initial state,
      normally 0. 
      \item The third \code{<term>} is a list of labels of the form
      \code{label(<term>,<int>)}, where \code{<term>} represents a production
      rule and \code{<int>} the label number. The production rule is represented
      as an ATerm and has, e.g., the form:
      \code{prod([sort("<START>"),char-class([256])],sort("<Start>"),no-attrs)}
      \item The states have the form: \code{state-rec(<int>,[<list>],[<list>])}
      where \code{<int>} represents the state number, the first list the list of
      gotos, and the second list the list of actions.
	\begin{itemize} 
      \item A goto item looks like \code{goto([<list>],<int>)}, this list is a
      list of single characters, character ranges, and/or label numbers. The
      \code{<int>} is the state number to which the parser has to jump. 
      \item An action item looks like \code{action([<list>],[<list>])}, where
      the first list is a list of single characters and/or character ranges.
      The second list are the actions that have to be performed:
	\begin{itemize} 
      \item \code{reduce(<int>,<int>,<int>)}: the first \code{<int>} represents
      the number of members in of the production rule with label given as the
      second \code{<int>}, and the last \code{<int>} represent that "status" of
      a production rule, \code{prefer}, \code{avoid}, \code{reject}, or
      \code{normal}. 
      \item \code{reduce(<int>,<int>,<int>,<term>)}: the \code{<term>}
      represents a lookahead set greater tha n one. Lookaheads of more than one
      character are performed dynamically. 
      \item \code{shift(<int>)}: the \code{<int>} represent the new state after
      shifting the current character.
      \item \code{accept}
    \end{itemize} 
    \end{itemize} 
	\item The priorities is a list of relations between 2 production rules: 
    \begin{itemize}
      \item \code{left-prio(<int>,<int>)}
      \item \code{right-prio(<int>,<int>)}
      \item \code{non-assoc-prio(<int>,<int>)}
      \item \code{gtr-prio(<int>,<int>)}
    \end{itemize}
	 where the
      \code{<int>}s are the labels of the production rules involved.
      \code{left}, \code{right}, and \code{non-assoc} are associativity
      attributes used within the SDF definition. \code{gtr} is the priority
      relation.
    \end{itemize} 

\end{document}

    