\documentclass{article}

	\usepackage{hevea}

	\title{Frequently Asked Questions about JSGLR}
	\author{Karl Trygve Kalleberg}
	
	\newcommand{\cc}[1]{\texttt{#1}}
	\newcommand{\tp}[1]{\texttt{#1}}

\begin{document}

\section{Frequently Asked Questions about JSGLR}

\subsection{What do I need to parse a file?}

	You need a valid parse table. This is generated from an SDF grammar using the
	\cc{sdf2table} tool. This tool is part of the SDF package downloadable from
	\ahref{http://syntax-definition.org}{syntax-definition.org}.
	
\subsection{How do I build a parse table?}

	Using the \cc{sdf2table} tool:
	
\begin{verbatim}
# pack-sdf -i lang.sdf -o lang.def
# sdf2table -m lang -i lang.def -o lang.tbl
\end{verbatim}

\subsection{How do I deal with syntax errors?}

	When the parser encounters an unexpected character (token), it throws
	either a \tp{BadTokenException} or a \tp{TokenExpectedException}. Both
	exceptions contain the token actually seen, along with the location (offset,
	line number, column number). The latter also contains the expected token. 
	
	Due to the nature of GLR parsing, multiple tokens may be expected (since
	multiple derivations may be possible). The present implementation will only
	return a \tp{TokenExpectedException} when exactly one token is expected,
	otherwise it will throw a \tp{BadTokenException}.

\subsection{Parsing is slow, how do I improve performance?}

	The current implementation is highly unoptimized. It does an excellent job of
	creating a lot of garbage during parsing. The good news is: it's not your
	fault, and we're fixing it. In the meantime, put the parser in another thread
	if you're developing an interactive application, and apply patience if not.
 
\subsection{How do I use my own tree factory?}

	This is currently not supported. The parser will always build parse trees on
	the AsFix2 form. 
	
\subsection{How do I deal with threading?}

	The parser implementation is not thread-safe. You need thread-safety, you
	must provide it.

\subsection{What do I need to run this thing?}

	The implementation relies on the Java ATerm library for representing the parse
	trees. It part of the binary distribution. A separate source tarball is
	available to accompany the source distribution of JSGLR. 
	
\subsection{Why isn't everything 100\% Java?}

	It is. Or, if you mean, why is the parser generator not also written in Java?
	Legacy reasons. It was originally developed before Java was invented. There
	are slow-moving projects in the works to make it available on the JVM.
	
\subsection{How do I generate my own tables?}
	 
	You're apparently a (wo)man looking for adventure. You must generate tables
	according to the \ahref{parse-table-format.html}{parse table format}. Note
	that while the tables generated by \cc{sdf2table} will be in an obscure binary
	format, JSGLR also understands tables written in text. Follow the examples in
	the parse table format reference.
	
\subsection{How does the \cc{table-info} tool work?}

	It's not written yet, so it doesn't.

\end{document}